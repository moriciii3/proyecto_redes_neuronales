\documentclass[12pt]{article}
\usepackage[utf8]{inputenc}
\usepackage[a4paper,margin=1in]{geometry}
\usepackage{hyperref}
\usepackage{enumitem}

\begin{document}
	
	% --- PORTADA ---
	\begin{titlepage}
		\centering
		\vspace*{4cm}
		
		{\LARGE \textbf{Proyecto INF395 \\[0.3cm]
				Introducción a las Redes Neuronales y Deep Learning\\[0.5cm]
				Entrega 0\\[3cm]}}
		
		\textbf{\large Integrantes:}\\
		\begin{itemize}[leftmargin=3cm]
			\item Bruno Morici \hspace{0.5cm} Rol: 202373555-8
			\item Juan Pablo Fuenzalida \hspace{0.5cm} Rol: 202373102-1
		\end{itemize}
		
		\vspace{1cm}
		\textbf{\large Ramo:}\\
		\begin{itemize}[leftmargin=3cm]
			\item Sigla: INF395
			\item Semestre: 2025-2
			\item Profesor: Alejandro Veloz
		\end{itemize}
		
		\vfill
		
	\end{titlepage}
	
	% --- CONTENIDO ---
	\section*{Título del tema}
	Clasificación y predicción de eventos de colisión electrónica en CERN usando aprendizaje profundo.
	
	\section*{Tipo de problema}
	Aprendizaje supervisado (clasificación y regresión).
	
	\section*{Descripción}
	El dataset contiene eventos de colisión de dos electrones, con variables como energía, momento, ángulos y masa invariante. El objetivo es predecir la masa invariante o clasificar eventos relevantes mediante redes neuronales.
	
	\section*{Problema a abordar por el grupo}
	Implementar y evaluar modelos de aprendizaje profundo que:
	\begin{itemize}
		\item Predigan la masa invariante (regresión).
		\item Comparen resultados con implementaciones existentes.
	\end{itemize}
	
	\section*{Punto de partida}
	Existe un repositorio y notebooks con aproximaciones iniciales:  
	\begin{itemize}
		\item \url{https://github.com/Isa1asN/cern-electron-collision-data-m-prediction}
		\item \url{https://www.kaggle.com/code/mesutssmn/cern-electron-collision-prediction}
	\end{itemize}
	
	\section*{Dataset(s)}
	Contiene 100 000 eventos con variables como:
	\begin{itemize}
		\item Energías (\texttt{E1}, \texttt{E2}), momento (\texttt{px, py, pz}), pseudorapidez (\texttt{eta}), ángulo phi (\texttt{phi}), cargas (\texttt{Q1, Q2}) y masa invariante (\texttt{M}).
	\end{itemize}
	
	Disponible en Kaggle: \url{https://www.kaggle.com/datasets/fedesoriano/cern-electron-collision-data}.
	
	\section*{Artículos relacionados}
	\begin{itemize}
		\item Andrews et al. (2018): Aplicación de clasificadores end-to-end basados en imagen para distinguir eventos de colisión en el LHC usando datos CMS (\url{https://arxiv.org/abs/1807.11916}).
		\item Kaggle Notebook: Implementación de redes neuronales para predicción de colisiones (\url{https://www.kaggle.com/code/mesutssmn/cern-electron-collision-prediction}).
	\end{itemize}
	
\end{document}

