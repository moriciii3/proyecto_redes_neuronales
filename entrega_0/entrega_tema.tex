\documentclass[12pt]{article}
\usepackage[utf8]{inputenc}
\usepackage[a4paper,margin=1in]{geometry}
\usepackage{hyperref}
\usepackage{enumitem}

\begin{document}
	
	% --- PORTADA ---
	\begin{titlepage}
		\centering
		\vspace*{4cm}
		
		{\LARGE \textbf{Entrega 0: Proyecto para ramo INF395 \\[0.3cm]
				Introducción a las Redes Neuronales y Deep Learning}}\\[3cm]
		
		\textbf{\large Integrantes:}\\
		\begin{itemize}[leftmargin=3cm]
			\item Bruno Morici \hspace{0.5cm} Rol: 202373555-8
			\item Juan Pablo Fuenzalida \hspace{0.5cm} Rol: 202373102-1
		\end{itemize}
		
		\vspace{1cm}
		\textbf{\large Ramo:}\\
		\begin{itemize}[leftmargin=3cm]
			\item Sigla: INF395
			\item Semestre: 2025-2
			\item Profesor: Alejandro Veloz
		\end{itemize}
		
		\vfill
		{\large \today}
		
	\end{titlepage}
	
	% --- CONTENIDO ---
	\section*{Título del tema}
	Clasificación de eventos de colisión electrónica en CERN usando aprendizaje profundo
	
	\section*{Tipo de problema}
	Aprendizaje supervisado – clasificación/regresión (predicción de masa invariante, clasificación de eventos físicos)
	
	\section*{Descripción}
	Se dispone de un conjunto de datos de colisiones de dos electrones (dielectron events) con características como energía, momento, cargas, ángulos y masa invariante. El dataset contiene alrededor de 100 000 eventos en el rango de masa invariante entre 2 y 110 GeV. El objetivo es utilizar técnicas de redes neuronales (por ejemplo, redes densas, CNN o GNN) para predecir la masa invariante o clasificar tipos de eventos a partir de las características disponibles.
	
	\section*{Problema a abordar por el grupo}
	Nuestro objetivo es implementar y evaluar modelos de aprendizaje profundo que puedan:
	\begin{itemize}
		\item Predecir la masa invariante (regresión) con mínima pérdida (ej. MSE).
		\item (Opcional) Clasificar eventos que podrían indicar procesos físicos relevantes.
		\item Comparar resultados con implementaciones existentes como la predicción de masa usando ANN (ver repositorio en GitHub).
	\end{itemize}
	
	\section*{Punto de partida}
	Existe un repositorio público en GitHub que contiene el dataset (\texttt{dielectron.csv}) y un notebook (\texttt{prediction.ipynb}) con una predicción inicial mediante redes neuronales: \url{https://github.com/Isa1asN/cern-electron-collision-data-m-prediction}. Nuestro trabajo se diferenciará al explorar:
	\begin{itemize}
		\item Arquitecturas más sofisticadas (CNN sobre representación gráfica de los eventos, o GNN).
		\item Ajuste de hiperparámetros y comparación de rendimiento.
		\item Análisis de importancia de características (feature importance).
	\end{itemize}
	
	\section*{Dataset(s)}
	El dataset contiene 100 000 eventos de colisión electrónica con las siguientes variables:
	\begin{itemize}
		\item Energías (\texttt{E1}, \texttt{E2})
		\item Componentes del momento (\texttt{px1}, \texttt{py1}, \texttt{pz1}, \texttt{px2}, \texttt{py2}, \texttt{pz2})
		\item Momento transverso (\texttt{pt1}, \texttt{pt2})
		\item Pseudorapidez (\texttt{eta1}, \texttt{eta2})
		\item Ángulo phi (\texttt{phi1}, \texttt{phi2})
		\item Carga (\texttt{Q1}, \texttt{Q2})
		\item Masa invariante (\texttt{M})
	\end{itemize}
	
	Disponible en Kaggle: \url{https://www.kaggle.com/datasets/fedesoriano/cern-electron-collision-data}. También hay un notebook de Kaggle que explora algoritmos de árbol de decisión: \url{https://www.kaggle.com/code/mariyagoliyad/dt-algorithms-for-cern-electron-collision-data}.
	
	\section*{Artículos relacionados}
	\begin{itemize}
		\item Andrews et al. (2018): Aplicación de clasificadores end-to-end basados en imagen para distinguir eventos de colisión en el LHC usando datos CMS (\url{https://arxiv.org/abs/1807.11916}).
		\item Fernández Madrazo et al. (2017): Aplicación de redes convolucionales para clasificar colisiones en física de alta energía transformando datos físicos en imágenes (\url{https://arxiv.org/abs/1708.07034}).
		\item Mikołaj Kita et al. (2024): Uso de modelos de difusión generativa para acelerar simulaciones de colisiones en experimentos de CERN (\url{https://arxiv.org/abs/2406.03233}).
		\item Verdone et al. (2024): Mejora del análisis de colisiones en física de partículas con Graph Neural Networks y técnicas de atribución de datos (\url{https://arxiv.org/abs/2407.14859}).
		\item Tesis de identificación de partículas en ALICE usando ML y modelos generativos (\url{https://alice-collaboration.web.cern.ch/node/35772}).
	\end{itemize}
	
\end{document}

